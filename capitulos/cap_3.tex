% ----------------------------------------------------------
\chapter{Preparação dos dados para o treinamento}
% ----------------------------------------------------------

Antes de iniciar o treinamento, exitem alguns procedimentos iniciais que garantem desde a coleta correta das informações necessárias para o modelo, até a limpeza, transformação e redução dos dados. Nesta seção, serão abordadas as etapas necessárias para preparar os dados para o treinamento do modelo, seguindo o fluxo de trabalho descrito na Figura \ref{fig:passos_preparacao}.

\begin{figure}[H]
	\caption{\label{fig:passos_preparacao}Passos para preparação dos dados.}
	\begin{center}
		\includegraphics[scale=0.4]{figuras/steps_data_preparing.png}
	\end{center}
	\fonte{\cite{pecan_data_prep_2023}}
\end{figure}

\section{Coleta de dados}

Considerando a premissa do trabalho, em que a previsão do nível do rio será dada a partir de dados meteorológicos da cidade de Porto Alegre, junto aos dados de monitoramento do nível dos rios que constituem a bacia do Guaíba, duas fonte de dados foram utilizadas. Para os dados meteorológicos, o portal do \textit{INMET}\footnote{\url{https://portal.inmet.gov.br/}} (Instituto Nacional de Meteorologia) foi utilizado, onde foram coletadas as informações de temperatura, umidade relativa do ar, precipitação e velocidade do vento. Já os dados de monitoramento dos rios foram coletados no site da \textit{SEMA-RS}\footnote{\url{https://www.saladesituacao.rs.gov.br/dados}} (Sala de situação), onde foram coletados os dados de nível do rio Guaíba,Caí, Jacuí, Sinos e Gravataí. Os dados meteorológicos foram coletados em formato \textit{CSV}, com arquivos separados em anos de monitoramento com frequência horária, enquanto os dados dos níveis dos rios foram coletados no formato \textit{.xlsx}, com histórico completo de amostragem das informações em frequência de 15 minutos, utilizando a biblioteca \textit{Pandas} do Python.

\section{Pré processamento dos dados}

Antes de seguir para o próximo passo mostrado na Figura \ref{fig:passos_preparacao}, os dados coletados necessitam de um pré-processamento específico para cada uma das fontes utilizadas. Para os dados meteorológicos, devido às informações estarem separadas por ano de monitoramento, foi necessário concatenar os arquivos de cada ano em um único arquivo, utilizando a função \textit{concat} da biblioteca \textit{Pandas}, removendo o cabeçalho de informações geográficas da estação, ilustrado na Figura \ref{fig:base_inmet} Além disso, foi necessário combinar as duas primeiras colunas e converter o formato de data e hora para o padrão \textit{datetime}, utilizando a função \textit{to\_datetime} da mesma biblioteca. 

\begin{figure}[H]
	\caption{\label{fig:base_inmet}Dados meteorológicos.}
	\begin{center}
		\includegraphics[scale=0.25]{figuras/base_inmet.png}
	\end{center}
	\fonte{Autor.}
\end{figure}

Analisando as colunas e os seus respectivos tipos de dados, tem-se a seguinte tabela:

\begin{table}[H]
	\centering
	\begin{tabular}{|p{10cm}|c|}
	\hline
	\textbf{Coluna} & \textbf{Tipo} \\
	\hline
	Precipitação Total (mm) & float64 \\
	Pressão Atmosférica ao Nível da Estação (mB) & float64 \\
	Pressão Atmosférica Máx. na Hora Anterior (mB) & float64 \\
	Pressão Atmosférica Mín. na Hora Anterior (mB) & float64 \\
	Radiação Global (kJ/m²) & float64 \\
	Temperatura do Ar - Bulbo Seco (°C) & float64 \\
	Temperatura do Ponto de Orvalho (°C) & object \\
	Temperatura Máxima na Hora Anterior (°C) & object \\
	Temperatura Mínima na Hora Anterior (°C) & object \\
	Temperatura Orvalho Máx. na Hora Anterior (°C) & float64 \\
	Temperatura Orvalho Mín. na Hora Anterior (°C) & float64 \\
	Umidade Relativa Máx. na Hora Anterior (\%) & float64 \\
	Umidade Relativa Mín. na Hora Anterior (\%) & object \\
	Umidade Relativa do Ar (\%) & object \\
	Vento - Direção Horária (° (gr)) & object \\
	Vento - Rajada Máxima (m/s) & float64 \\
	Vento - Velocidade Horária (m/s) & float64 \\
	\hline
	\end{tabular}
	\caption{Tabela de tipos de dados da base de informações meteorológicas.}
	\label{tab:colunas_dados_meteorologicos}
\end{table}

A partir dela, nota-se que algumas colunas estão com o tipo de dado \textit{object}, o que indica que os dados não estão no formato correto. Para resolver isso, foi necessário converter as colunas de temperatura e umidade relativa do ar para o tipo \textit{float64}.

Já para os dados dos níveis dos rios, também foi necessário remover o cabeçalho com dados geográficos da estação de monitoramento, como mostra a Figura \ref{fig:base_sema}, junto da conversão do formato de data e hora para o padrão \textit{datetime} da primeira coluna.

\begin{figure}[H]
	\caption{\label{fig:base_sema}Dados meteorológicos.}
	\begin{center}
		\includegraphics[scale=0.5]{figuras/base_sema.png}
	\end{center}
	\fonte{Autor.}
\end{figure}

\section{Limpeza dos dados}
\label{sec:limpeza_dos_dados}

Após a coleta dos dados, o próximo passo é a limpeza dessas informações, cujo procedimento consiste em remover dados duplicados, corrigir erros de formatação e lidar com valores ausentes. Existem diferentes abordagens para tratar valores inconsistentes no dados coletados. Uma abordagem comum, principalmente em casos onde não há uma linearidade ou tendência clara é o preenchimento com zero, como mostra a Figura \ref{fig:passo_dados_limpeza}, ou com a média dos dados na coluna a ser limpa.

\begin{figure}[H]
	\caption{\label{fig:passo_dados_limpeza}Limpeza dos dados coletados.}
	\begin{center}
		\includegraphics[scale=0.4]{figuras/step_data_cleaning.png}
	\end{center}
	\fonte{\cite{pecan_data_prep_2023}}
\end{figure}

Nessa etapa, na base de dados meteorológicos, optou-se por preencher os dados ausentes, representados por ''-'', e os dados com valor igual a ''-9999'' por zero, já que dados meteorológicos tendem a ser mais voláteis e não apresentam uma tendência clara. Desse modo, a abordagem de aproximação linear é descartada, a fim de não comprometer a análise do modelo de previsão.  

Tomando como exemplo a coluna \textit{TEMPERATURA ORVALHO MAX. NA HORA ANT. (AUT) (°C)}, listada na Tabela \ref{tab:colunas_dados_meteorologicos}, na Figura \ref{fig:dados_clima_poa_nao_tratados}, observa-se os valores ausentes representados por ''-'' e ''-9999''.

\begin{figure}[H]
	\caption{\label{fig:dados_clima_poa_nao_tratados}Gráfico de Temperatura do orvalho Máx. na Hora Anterior (°C) não tratado.}
	\begin{center}
		\includegraphics[scale=0.35]{figuras/TEMPERATURA ORVALHO MAX. NA HORA ANT. (AUT) (°C) SEM TRATAMENTO.png}
	\end{center}
	\fonte{Autor.}
\end{figure}

Após a substituição dos valores ausentes por zero, o gráfico da mesma coluna, mostrado na Figura \ref{fig:dados_clima_poa_tratados}, apresenta uma distribuição mais uniforme, sem os picos de dados ausentes.

\begin{figure}[H]
	\caption{\label{fig:dados_clima_poa_tratados}Gráfico de Temperatura do orvalho Máx. na Hora Anterior (°C) tratado.}
	\begin{center}
		\includegraphics[scale=0.35]{figuras/TEMPERATURA ORVALHO MAX. NA HORA ANT. (AUT) (°C) COM TRATAMENTO.png}
	\end{center}
	\fonte{Autor.}
\end{figure}

Aplicado o tratamento, a coluna \textit{TEMPERATURA ORVALHO MAX. NA HORA ANT. (AUT) (°C)} passa a apresentar uma distribuição mais uniforme, sem os picos de dados ausentes, com valores condizentes com a escala esperada da unidade medida, neste caso, a temperatura em graus Celsius.

Em casos onde os dados apresentam \textit{outliers} (dados que se distanciam significativamente do restante da coluna analisada), foi aplicado o método IQR (Interquartile Range) para identificar e remover esses valores. A partir da diferença entre o terceiro quartil (Q3) e o primeiro quartil (Q1) de um conjunto de dados, multiplicado por um fator de tolerância, o método determina limites superiores e inferiores para o conjunto de dados analisado, removendo ou substituindo os valores que estão fora dessa faixa, para manter a consistência das informações.

Por exemplo, na coluna \textit{UMIDADE REL. MAX. NA HORA ANT. (AUT) (\%)}, as Figuras \ref{fig:dados_outlier_sem_tratamento} e \ref{fig:dados_outlier_com_tratamento} mostram a diferença entre os dados após a aplicação do método IQR, com um fator de tolerância de 1.2 entre os interquartis Q1 e Q3.

\begin{figure}[H]
	\caption{\label{fig:dados_outlier_sem_tratamento}Gráfico de Umidade Relativa Máxima  na Hora Anterior (\%) sem tratamento.}
	\begin{center}
		\includegraphics[scale=0.35]{figuras/UMIDADE REL. MAX. NA HORA ANT. (AUT) SEM TRATAMENTO.png}
	\end{center}
	\fonte{Autor.}
\end{figure}

\begin{figure}[H]
	\caption{\label{fig:dados_outlier_com_tratamento}Gráfico de Umidade Relativa Máxima  na Hora Anterior (\%) com tratamento.}
	\begin{center}
		\includegraphics[scale=0.35]{figuras/UMIDADE REL. MAX. NA HORA ANT. (AUT) COM TRATAMENTO.png}
	\end{center}
	\fonte{Autor.}
\end{figure}

Na base de dados dos níveis dos rios, embora a dinâmica que representa os respectivos comportamentos não seja linear, o uso do conceito de aproximação linear permite preencher os dados ausentes por meio da interpolação entre os valores que cercam as células sem informação em determinado período. Essa abordagem é válida, uma vez que a variação do nível do rio entre dois pontos de amostragem próximos tende a ser linear, já que o nível do rio não apresenta oscilações bruscas em curtos períodos de tempo.

Para realizar a interpolação, utilizou-se a função \textit{interpolate} com o argumento \textit{method = linear} da biblioteca \textit{Pandas}, que preenche os valores ausentes com base nos dados adjacentes.

Ainda, observou-se nas bases dos níveis dos rios a ocorrência de valores ausentes no início ou no final do período de amostragem, inviabilizando a interpolação. Para esses casos, o preenchimento foi feito a partir da repetição do primeiro ou do último valor disponível, respectivamente. É importante ressaltar que tal limpeza só pode ser feita se o preenchimento ocorra em um intervalo de tempo curto, visando não afetar a análise feita pelo modelo de previsão.

\section{Redução dos dados}

A etapa de redução dos dados visa eliminar informações redundantes ou irrelevantes, mantendo apenas os dados que contribuem para a análise e treinamento do modelo de previsão. Considerando que serão utilizados dados de monitoramento do nível de 4 rios para a previsão do nível do rio Guaíba, os dados meteorológicos coletados precisam ser filtrados de modo a garantir tanto que as informações sejam relevantes para a previsão, quanto que os dados de monitoramento dos rios estejam alinhados com os dados meteorológicos, evitando assim a inclusão de dados desnecessários no treinamento. 

Voltando para a Tabela \ref{tab:colunas_dados_meteorologicos}, nota-se a presença de 17 colunas, das quais:

\begin{itemize}
	\item 6 colunas referem-se a dados de temperatura;
	\item 3 colunas referem-se a dados de pressão atmosférica;
	\item 3 colunas referem-se a dados do comportamento do vento;
	\item 3 colunas referem-se a dados de umidade relativa do ar;
	\item 1 coluna refere-se a dados de radiação solar.
	\item 1 coluna refere-se a dados de precipitação.
\end{itemize}

Desse modo, para o primeiro passo de redução da base, foram mantidas apenas uma coluna de cada tipo de dado, filtrando 6 colunas no total, conforme a Tabela \ref{tab:colunas_dados_meteorologicos_reduzidas}. 

\begin{table}[H]
	\centering
	\begin{tabular}{|p{10cm}|c|}
	\hline
	\textbf{Coluna} & \textbf{Unidade de medida} \\
	\hline
	Temperatura do Ar - Bulboeco & (°C) \\
	Pressão Atmosférica ao Nível da Estação & (mB) \\
	Vento - Velocidade Horária & (m/s) \\
	Umidade Relativa do Ar & (\%) \\
	Radiação Global & (kJ/m²) \\
	Precipitação Total & (mm) \\
	\hline
	\end{tabular}
	\caption{Tabela de dados meteorológicos reduzidos - 1ª Filtragem}
	\label{tab:colunas_dados_meteorologicos_reduzidas}
\end{table}

Para a escolha das colunas a serem mantidas, foram levados em consideração os seguintes critérios:
\begin{itemize}
	\item As colunas \textit{Temperatura do Ar - Bulboeco}, \textit{Pressão Atmosférica ao Nível da Estação} e \textit{Umidade Relativa do Ar} foram escolhidas por serem medidas diretas dos respectivos fenômenos atmosféricos, enquanto as demais colunas referem-se a medidas derivadas ou não diretamente observáveis, que não são tão relevantes para a previsão do nível do rio.
	\item A coluna \textit{Precipitação Total} foi mantida por ser um dos principais fatores que influenciam o nível do rio.
	\item As colunas \textit{Radiação Global} e \textit{Umidade Relativa do Ar} foram mantidas por serem fatores importantes para a evaporação da água, que também influenciam, mesmo que de forma indireta, no nível do rio.
\end{itemize}

Após essa redução inicial, os gráficos e seus dados de cada coluna foram analisados, a fim de verificar se as informações eram consistentes e poderiam contribuir para o treinamento do modelo. Para isso, foram utilizados gráficos de dispersão e histogramas, como os mostrados na Figura \ref{fig:histograma_dados_meteorologicos}.

ADICIONAR AQUI O HISTOGRAMA DOS DADOS METEOROLÓGICOS E A DECISÃO DE QUAIS COLUNAS FORAM MANTIDAS E QUAIS FORAM DESCARTADAS.

Definidas as colunas a serem mantidas, o próximo passo consistiu em alinhar os dados meteorológicos com os dados de monitoramento dos rios, garantindo que as informações estivessem na mesma frequência de amostragem. Para isso, foram utilizados os dados de monitoramento do nível do rio Guaíba, que possuem a maior frequência de amostragem entre os rios analisados, com intervalo de 15 minutos, enquanto os dados meteorológicos possuem frequência de 1 hora (60 minutos). Assim, a frequência de amostragem das informações do nível dos rios foi reduzida para 1 hora, aplicando um filtro simples que mantém apnas as linhas cujo \textit{timestamp} tem minuto igual a 0.

Por fim, alinhados os dados meteorológicos com os dados de monitoramento dos rios, o \textit{dataframe} resultante passou um nivelamento superior e inferior quanto a data inicial e final, de modo que o período de amostragem de todas as informações fosse o mesmo, garantindo que todas as colunas tivessem o mesmo número de linhas preenchidas. Embora na etapa de limpeza dos dados tenha sido aplicado o preenchimento de valores ausentes, verificou-se que as estações de monitoramento dos rios não possuem medições que se iniciam ou finalizam na mesma data. Dessa forma, a fim de analisar qual o período de início de fim das informações obtidas, a tabela \ref{tab:periodo_amostragem} apresenta o período de amostragem de cada uma das fontes de dados.
\begin{table}[H]
	\centering
	\begin{tabular}{|l|c|c|c|c|}
	\hline
	\textbf{Rio} & \textbf{Data Mínima} & \textbf{Hora Mínima} & \textbf{Data Máxima} & \textbf{Hora Máxima} \\
	\hline
	Rio dos Sinos & 2013-12-13 & 05:00:00 & 2024-06-30 & 09:00:00 \\
	Rio Caí       & 2015-09-14 & 13:00:00 & 2024-05-06 & 14:00:00 \\
	Rio Gravataí  & 2017-11-07 & 12:00:00 & 2024-07-01 & 00:00:00 \\
	Rio Jacuí     & 2014-10-08 & 20:00:00 & 2024-04-27 & 01:00:00 \\
	Rio Guaíba    & 2014-07-29 & 14:00:00 & 2024-05-06 & 14:00:00 \\
	\hline
	\end{tabular}
	\caption{Datas mínima e máxima disponíveis para cada rio analisado}
	\label{tab:periodo_amostragem}
\end{table}

Quanto a base de dados meteorológicos, o site do INMET disponibiliza dados desde o início dos anos 2000 e mantém a base atualizada até os dias atuais, não sendo, portanto, um limitador para a definição do período do dataframe final. Assim, o período de amostragem final pode ser definido a partir de 07 de novembro de 2017, que é a data mínima do rio Gravataí, até 06 de maio de 2024, que é a data máxima do rio Caí, com um dataframe de 56366 linhas.

\section{Transformação dos dados}

A transformação dos dados é uma etapa fundamental para garantir que as informações estejam no formato correto para o treinamento do modelo de previsão. Nesta fase, os dados são convertidos em um formato numérico, adequado para algoritmos de aprendizado de máquina, e normalizados para assegurar que todas as variáveis tenham a mesma escala.

Com os dados meteorológicos e de monitoramento dos rios alinhados, a normalização dos dados é aplicada visando principalmente equilibrar os dados dos níveis dos rios com os dados meteorológicos, tendo em vista que as escalas entre essas informações apresentam uma discrepância maior, pela diferença de unidade de medida entre elas.

Ademais, outro fator que deve ser considerado é se as colunas do \textit{dataframe} são compostas de dados categóricos (geralmente representados através de textos) ou numéricos. Por se tratar de medições aferidas por sensores meteorológicos e/ou geográficos, todas as colunas da base de dados estudada são numéricas.

A partir dessa premissa, a normalização dos dados foi feita utilizando o método \textit{StandardScaler} da biblioteca \textit{Scikit-learn}, que transforma os dados de uma coluna para um valor cuja média seja zero e desvio padrão igual a um. A pontuação padrão de uma amostra x é dada por:

\begin{equation}
z = \frac{x - \mu}{\sigma}
\end{equation}

onde \( \mu \) é a média da amostra e \( \sigma \) é o desvio padrão. Muitos elementos usados em funções objetivas de um algoritmo de aprendizagem (como o kernel RBF do Support Vector e as máquinas ou os regularizadores L1 e L2 dos modelos lineares) assumem que todos os recursos estão centrados em torno uma média igual a zero e variação de mesma ordem \cite{scikit_learn_standardscaler}. Desse modo, caso uma amostra apresente uma variância de magnitude maior do que outros, ele pode dominar a função objetivo e fazer o estimador incapaz de aprender com outros recursos corretamente como esperado.

Além disso, o método aplicado também é sensível a outliers, que afetam a média e o desvio padrão calculados para a definição dos valores normalizados. Por esse motivo, na seção \ref{sec:limpeza_dos_dados}, a aplicação do método IQR para remoção dos \textit{outliers}, além de limpar os dados durante aquela etapa de preparação, garantiu que os dados estivessem mais homogêneos e não fossem distorcidos por valores extremos na normalização desta etapa.

Para exemplificar a normalização dos dados, 

Com os dados normalizados, o \textit{dataframe} está pronto para ser dividido em conjuntos de treinamento e teste, chegando ao estágio final de preparação, e seguindo para o treinamento do modelo de previsão. 

\section{Divisão dos dados em conjuntos de treinamento e teste}





