% ----------------------------------------------------------
\chapter{Fundamentação Teórica}\label{cap:fundamentacaoTeorica}
% ----------------------------------------------------------
Explicar brevemente o que será tratado como fundamentação teórica para o entendimento do contexto em que o modelo de aprendizagem de máquina será aplicado.

\section{As mudanças climáticas e as catástrofes natuais}

As grandes cidades brasileiras enfrentam desafios cada vez mais frequentes e crescentes devido às mudanças climáticas, que agravam problemas como enchentes, inundações e deslizamentos. Projeções indicam que, até 2030, a mancha urbana de São Paulo pode aumentar em até 38\%, ampliando o risco para mais de 20\% das áreas de expansão urbana, que se tornarão suscetíveis a acidentes naturais (Nobre et al., 2011). O estudo também destaca que o aumento na frequência de eventos de chuvas intensas pode dobrar o número de dias com precipitação acima de 10 milímetros, agravando a vulnerabilidade da população, especialmente nas áreas periféricas e de menor infraestrutura.

\section{Cenário de enchentes no sul do Brasil}

Com base no histórico das enchentes no Rio Grande do Sul, observa-se que os desastres relacionados ao excesso de chuvas não são um fenômeno recente. Desde 1941, o estado lida com eventos catastróficos, como a enchente que devastou Porto Alegre naquele ano, considerada uma das mais graves da história da cidade. Ao longo das décadas, esses episódios continuaram a ocorrer, expondo a vulnerabilidade da região diante de chuvas intensas e repentinas. A combinação de fatores naturais, como a geografia da região e os ciclos climáticos, aliado as ações humanas mais nocivas ao meio ambiente, contribui para a repetição e intensificação dessas tragédias (VEJA, 2024). 

Em Santa Catarina, estado adjacente ao Rio Grande do Sul, as enchentes também são fenômenos recorrentes que, ao longo dos anos, têm causado impactos sociais, econômicos e ambientais. Um dos eventos mais recentes foi registrado em maio de 2024, quando o estado enfrentou um dos dias mais chuvosos da história, levando ao transbordamento de rios, deslizamentos de terra e bloqueios em diversas rodovias. De acordo com reportagens da época, a quantidade excessiva de chuva foi um dos principais fatores que contribuíram para a gravidade da enchente, com destaque para a queda de barreiras e o isolamento de algumas regiões do estado (G1, 2024).

O Rio Guaíba, principal manancial de abastecimento de água para a capital do Rio Grande do Sul e região, é alvo de estudo sobre diversos temas, incluindo sua hidrodinâmica e nível ao longo do ano. No Artigo conduzido pelos pesquisadores Andrade \textit{et al}., a variabilidade nas descargas líquidas do Rio Guaíba revelou flutuações significativas nos volumes de descarga, variando de 407 m³/s a 14.270 m³/s, o que indica uma grande influência das condições climáticas sazonais e da vazão dos rios tributários, como o Jacuí, Taquarí, Caí e Sinos. Essas variações extremas foram observadas durante o período de 2014 a 2017 e reforçam a importância de monitorar continuamente o regime de águas do Guaíba para prevenir enchentes e outros desastres associados (Andrade et al., 2017).