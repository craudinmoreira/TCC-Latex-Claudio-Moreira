% ----------------------------------------------------------
\chapter{Conclusão}
% ----------------------------------------------------------
O presente trabalho teve como objetivo geral desenvolver um modelo de previsão do nível do Lago Guaíba, utilizando dados meteorológicos por meio da técnica de Regressão Ridge. Para alcançar esse propósito, foram estabelecidos objetivos específicos que envolveram desde a coleta e pré-processamento dos dados, até a implementação do modelo e análise dos resultados.

No desenvolvimento do projeto, procedeu-se à coleta de dados meteorológicos disponibilizados pelo INMET, assim como dos níveis hidrométricos dos rios da bacia do Guaíba, através do SEMA-RS. Esses dados passaram por um pré-processamento, que incluiu a limpeza, tratamento de valores ausentes, remoção de outliers e normalização, garantindo maior qualidade ao conjunto utilizado para treinamento. Foi também realizada a redução da dimensionalidade, mantendo apenas variáveis com maior relevância para a previsão, como temperatura do ar, pressão atmosférica, radiação global, velocidade do vento, umidade relativa e precipitação.

Na etapa de implementação, o modelo de Regressão Ridge foi escolhido por sua capacidade de lidar com multicolinearidade e reduzir o risco de sobreajuste, característica importante considerando a volatilidade das variáveis climáticas. Foram testados diferentes valores do parâmetro de regularização (\texttt{alpha}), variando também as proporções de divisão entre conjuntos de treinamento e teste (80:20, 70:30, 60:40). Para avaliar o desempenho da previsão do modelo, as métricas de desempenho — MSE, RMSE, MAE e R² foram utilizadas, visando uma análise mais objetiva e estatística da implementação.

Os resultados evidenciaram a robustez do modelo e a correlação significativa entre variáveis meteorológicas e o nível do Lago Guaíba. Apesar de pequenas variações nos indicadores de desempenho, não se observou impacto expressivo da alteração dos valores de alpha, sugerindo que o modelo atinge um platô de desempenho para os intervalos testados. Tal constatação indica que ajustes adicionais de regularização, ou a inclusão de novas variáveis externas, poderiam ser explorados em trabalhos futuros para ganhos incrementais de performance. Além disso, cabe a comparação com outros modelos de previsão existentes, ou até mesmo uma rede neural para comparar o desempenho em relação ao modelo Ridge.

Assim, conclui-se que o objetivo proposto foi atingido. O modelo desenvolvido mostrou-se eficiente e tecnicamente sólido para prever o nível do Lago Guaíba a partir dos dados meteorológicos utilizados. Entretanto, vale ressaltar que os dados utilizados são históricos, ou seja, compõem medidas meteorológicas aferidas por instrumentos no momento da medição. Para uma aplicação real de previsão do nível do rio, visando um alerta antecipado de enchente, o modelo seria expostos a medidas meteorológicas previstas ao invés de medidas, adicionando um grau de incerteza que pode gerar previsões com um grau de erro maior. Por isso, é necessário realizar testes com dados previstos, avaliar novas métricas de desempenho, para então considerar o uso do modelo em casos reais de previsão do nível do rio.  