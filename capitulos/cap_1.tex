% ----------------------------------------------------------
\chapter{Introdução}
% ----------------------------------------------------------
As mudanças climáticas têm intensificado a frequência e a severidade de eventos climáticos extremos, como chuvas intensas, secas prolongadas e inundações, impactando diretamente grandes centros urbanos e áreas rurais \cite{veja2024}. No Brasil, esses fenômenos têm se tornado mais frequentes, afetando a região Sul com inundações causadas por chuvas intensas que representam um desafio recorrente. As enchentes de 2024, que devastaram diversas cidades gaúchas, evidenciaram a necessidade de ferramentas eficazes para prever e mitigar os impactos de desastres naturais.

O problema abordado neste trabalho é a previsão do nível do Lago Guaíba, um dos principais mananciais de abastecimento de água de Porto Alegre e região metropolitana. \hl{No contexto do Lago Guaíba, a problemática é particularmente complexa devido à sua sensibilidade a chuvas intensas, tanto em seu leito principal quanto em seus afluentes, como os rios Jacuí, Caí, Gravataí e Sinos. As cheias de 2024 destacaram como a variabilidade climática e as contribuições dos afluentes amplificam as flutuações no nível do Guaíba, tornando essencial o desenvolvimento de modelos preditivos robustos} \cite{veja2024}.

A capacidade de prever o nível do rio é crucial para antecipar eventos de inundação, que podem causar perdas materiais significativas, deslocamento de populações e até mesmo mortes. A previsão do nível do rio, utilizando dados meteorológicos como variáveis preditoras, permite a elaboração de planos de evacuação, a alocação eficiente de recursos para mitigação de desastres e a proteção de infraestruturas críticas, contribuindo para a segurança e o bem-estar da população \cite{andrade2017}.

\hl{A importância de prever o nível de corpos d'água em geral, e do Lago Guaíba em Particular,} reside na possibilidade de reduzir os impactos socioeconômicos e ambientais causados pelas cheias. As inundações em Porto Alegre, como as observadas em 1941 e 2024, demonstram a vulnerabilidade da região a eventos climáticos extremos \cite{veja2024}. A elaboração de um modelo de previsão do nível do lago pode embasar decisões de políticas públicas e defesa civil, além de otimizar a gestão de recursos hídricos e outras aplicações que tangem a preservação e uso consciente do lago.

Apesar dos avanços em modelos hidrológicos e de previsão, as soluções atuais apresentam limitações que justificam a busca por novas abordagens. Algumas delas são computacionalmente custosas ou requerem dados extensivos que nem sempre estão disponíveis, dificultando sua implementação em tempo real \cite{andrade2017}.

Este trabalho propõe o uso de dados meteorológicos, como precipitação, temperatura, umidade e pressão atmosférica, combinados com técnicas de aprendizado de máquina, para prever o nível do rio, oferecendo uma ferramenta para a gestão de riscos de inundações.

\section{Objetivos}

O objetivo geral deste trabalho é desenvolver um modelo de previsão do nível do Lago Guaíba utilizando dados meteorológicos por meio de técnicas de aprendizado de máquina, especificamente a regressão Ridge.

Os objetivos específicos são:

\begin{itemize}
    \item Coletar e tratar dados meteorológicos e de níveis de rios relevantes para o modelo.
    \item Aplicar técnicas de pré-processamento de dados, incluindo limpeza e redução de dimensionalidade, para garantir a qualidade das informações.
    \item Implementar e treinar um modelo de regressão Ridge para prever o nível do Lago Guaíba com base nos dados meteorológicos.
    \item Avaliar o desempenho do modelo utilizando métricas de erro como MSE, RMSE, MAE e R².
    \item Analisar os resultados obtidos e propor ajustes para otimizar a previsão.
\end{itemize}

\section{Estrutura do trabalho}

\hl{O trabalho é constituído por 5 capítulos. No Capítulo 1 é apresentada a introdução, objetivos e estrutura do trabalho.}

\hl{O Capítulo 2 apresenta a fundamentação teórica sobre mudanças climáticas, catástrofes naturais, aprendizado de máquina e técnicas de regressão, incluindo conceitos de regressão linear e o modelo Ridge.}

\hl{No Capítulo 3 é apresentado o desenvolvimento, com ênfase na coleta de dados, pré-processamento, limpeza, redução, transformação e aplicação do modelo Ridge de previsão via \textit{script} na linguagem de programação Python.}

\hl{No Capítulo 4 são apresentados os resultados obtidos, incluindo gráficos de previsão, tabelas de métricas de desempenho para diferentes valores de alpha e proporções de divisão de dados, além de análise comparativa com estudos semelhantes. O Capítulo 5 trata da conclusão e discussões acerca do tema abordado e implementação desenvolvidaAo fim do trabalho encontram-se as referências.}

\hl{Dado que o presente trabalho envolve o desenvolvimento de um modelo preditivo baseado em dados meteorológicos e hidrológicos, os conceitos teóricos e metodológicos foram apresentados de forma prévia aos resultados. Consequentemente, as etapas de preparação de dados e implementação concentraram-se no desenvolvimento, com os resultados focados na avaliação do desempenho do modelo.}