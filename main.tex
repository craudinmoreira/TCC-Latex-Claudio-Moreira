% ------------------------------------------------------------------------
% ------------------------------------------------------------------------
% Modelo de TCC do curso de Engenharia de Controle e Automação - UFSC/Campus Blumenau
% Autor: Ciro André Pitz
% Revisão: Brenda Teresa Porto de Matos
% O presente modelo foi obtido a partir do modelo desenvolvido por Alisson Lopes Furlani disponível na BU/UFSC.
% ------------------------------------------------------------------------
% ------------------------------------------------------------------------

\documentclass[
12pt,				% tamanho da fonte
%openright,			% capítulos começam em pág ímpar (insere página vazia caso preciso)
oneside,			% para impressão no anverso. Oposto a twoside
a4paper,			% tamanho do papel. 
chapter=TITLE,		% títulos de capítulos convertidos em letras maiúsculas
section=TITLE,		% títulos de seções convertidos em letras maiúsculas
%subsection=TITLE,	% títulos de subseções convertidos em letras maiúsculas
%subsubsection=TITLE,% títulos de subsubseções convertidos em letras maiúsculas
% -- opções do pacote babel --
english,			% idioma adicional para hifenização
brazil				% o último idioma é o principal do documento
]{abntex2}

\usepackage{configs/eca_ufsc_bnu} % personalização da ABNTEX2 
\usepackage{float}
\addbibresource{pos_textual/referencias.bib} % Seus arquivos de referências

%---------------------------------------------------------------------------------------------
%--------- DADOS BÁSICOS DO TCC (Preencher todos) -------------------------------------
%---------------------------------------------------------------------------------------------
%Substituir 'Nome completo do autor' pelo seu nome.
\autor{Cláudio Lourenço Moreira}
% FIXME Substituir 'Título do trabalho' pelo título da trabalho.
\titulo{Previsão do Nível do Rio Guaíba a partir de Dados Meteorológicos}
% Substituir 'Subtítulo (se houver)' pelo subtítulo da trabalho. 
% Se não houver subtítulo, basta deletar o texto.
\subtitulo{Aplicação de Técnicas de Aprendizado de Máquina com Regressão Ridge}
% Substituir 'Orientador' pelo nome do seu orientador.
\orientador{Prof. Dr. Maiquel de Brito}
% Se for orientado por uma mulher, comente a linha acima e descomente a linha a seguir.
% \orientador[Orientadora]{Orientadora, Dra.}
% Substituir 'XXXXXX' pelo nome do seu  coorientador. Caso não tenha coorientador, comente a linha a seguir.
%\coorientador{Prof. Coorientador, Dr.}
% Se for coorientado por uma mulher, comente a linha acima e descomente a linha a seguir.
% \coorientador[Coorientadora]{Coorientadora, Dra.}
\dia{dia}
\mes{julho}
\ano{2025}
\local{Blumenau}
\formacao{Engenheiro de Controle e Automação}
% Se for mulher, comente a linha acima e descomente a linha a seguir.
%\formacao{Engenheira de Controle e Automação}
\bancaa{Prof. Dr. Maiquel de Brito} %Primeiro membro da banca (normalmente o orientador).
\bancab{Prof. Segundo, Dr.} %Segundo membro da banca
\bancac{Prof. Terceiro, Dr.} %Terceiro membro da banca

%Resumo e palavras-chave do trabalho
\resumotcc{No resumo são ressaltados o objetivo da pesquisa, o método utilizado, as discussões e os resultados com destaque apenas para os pontos principais. O resumo deve ser significativo, composto de uma sequência de frases concisas, afirmativas, e não de uma enumeração de tópicos. Não deve conter citações. Deve usar o verbo na voz ativa e na terceira pessoa do singular. O texto do resumo deve ser digitado, em um único bloco, sem espaço de parágrafo. O espaçamento entre linhas é simples e o tamanho da fonte é 12. Abaixo do resumo, informar as palavras-chave (palavras ou expressões significativas retiradas do texto) ou, termos retirados de thesaurus da área. Deve conter de 150 a 500 palavras. O resumo é elaborado de acordo com a NBR 6028.}
\palavraschave{palavra-chave 1; palavra-chave 2; palavra-chave 3.}

%Abstract e keywords do trabalho
\abstracttcc{Resumo traduzido para outros idiomas, neste caso, inglês. Segue o formato do resumo feito na língua vernácula. As palavras-chave traduzidas, versão em língua estrangeira, são colocadas abaixo do texto precedidas pela expressão ``Keywords'', separadas por ponto e vírgula.}
\keywords{keyword 1; keyword 2; keyword 3.}

%Agradecimentos (opcional). Caso não queira inserir, deixe em branco (\agradecimentostcc{} )
\agradecimentostcc{Ao longo dessa jornada árdua, desafiadora, porém imensamente prazerosa de se viver, diversas pessoas foram importantes para a minha formação como engenheiro, pessoa e profissional. Não poderia começar diferente meu agradecimento, senão aos meus pais, que são minha base, minha referência e minha motivação para sempre buscar ser melhor em todos os sentidos. Dar orgulho a eles é o que me move, e sem eles eu não teria sequer começado essa jornada. 

Na mesma prateleira de troféus e presentes dessa vida, estão meus amigos, que tenho o prazer e a alegria de dizer que ultrapassaram a barreira da universidade e se tornaram meus amigos pra vida, ou melhor, uma segunda família. A turma 19.1 foi especial desde o primeiro dia, e ao longo dos semestres, pandemia e demais percausos, essa turma se manteve unida e compartilhando dúvidas, risadas, histórias e momentos que guardo com imenso carinho. Meu obrigado a todos da turma 19.1, em especial ao time que me acompanhou um passo a mais de perto, e caminhou comigo dentro e fora da sala de aula. Vítor, Augusto, Samuel, Felipe (Borto) e Lauro, os Business Boys, a melhor equipe de robótica, a melhor turma de visão, de controle, de sistemas computacionais, ou de qualquer disciplina do curso, modéstia a parte. Vocês me mostraram o que é ser um estudante e um profissional de extrema qualidade, e eu sou muito grato pela convivência e pelos ensinamentos diários que tive perto de vocês.

Não seria justo deixar de dedicar um parágrafo separado ao professor que caminhou comigo como aluno, amigo e atualmente, família, João Victor Zanoni. Desde o primeiro dia de aula, você esteve ao meu lado como uma das maiores referências que tive na graduação. A tua determinação, dedicação, qualidade, profissionalismo e "n" outras virtudes fizeram do curso uma formação ímpar, sem sombra de dúvidas a melhor e mais especial que eu poderia ter. A UFSC, embora seja uma universidade de excelência, jamais teria me proporcionado uma formação tão completa e especial se não fosse por você. Cada trabalho, seja ele em dupla, trio, quarteto ou qualquer tamanho que fosse, eu sabia que seria no mínimo, eu e você. Foi uma honra vivenciar essa etapa da vida e todos esses obstáculos contigo, parceiro. Obrigado por tudo, professor Zanoni.

Aos meus amigos que tive a felicidade de conhecer fora da turma ao qual entrei, principalmente através da Integre Jr., meu muito obrigado. Passar pelo MEJ com vocês me trouxe um crescimento pessoal e profissional que nenhuma disciplina poderia agregar, e tudo só se tornou mais especial porque esssa empresa estava formada por pessoas como vocês. Agradeço por cada evento, cada reunião, desafios, projetos, imersões, viagens e momentos que cultivamos juntos.

Por fim, meu agradecimento a todos os professores que passaram pela minha vida acadêmica, e que contribuíram de alguma forma para a minha formação. o Campus de Blumenau pode ser pequeno em tamanho, mas em compensação tem uma equipe de professores extremamente qualificados e com grande gabarito, que tornanram a transmissão de conhecimento algo rico e determinante na minha formação. Em especial, agradeço ao meu professor orientador, Maiquel de Brito, que me acompanhou nesse trabalho e me deu todo o suporte necessário nessa reta final, além de estar presente desde o meu primeiro ano no curso.

É difícil agradecer a todos que contribuíram de alguma forma nessa metade de década da minha vida, acredito que cada pessoa que passar por nós e deixa uma marca, não importa o tamanho, é importante para chegarmos aonde estamos agora, e por isso, agradeço de coração a todos que fizeram parte de capítulo tão transformador.}

%Epígrafe (opcional). Caso não queira inserir, deixe em branco (\epigrafetcc{} )
\epigrafetcc{"A única maneira de se definir o limite do possível é ir além dele, para o impossível." (CLARKE, 1962)}

%Decatória (opcional). Caso não queira inserir, deixe em branco (\dedicatoriatcc{} )
\dedicatoriatcc{Dedido este trabalho aos meus pais, amigos, professores, e a todos que passaram e deixaram uma marca, seja ela qual for, ao longo desse capítulo da graduação.}

%Lista de quadros
%Além de figuras e tabelas, o TCC contém quadros? Caso afirmativo digite sim ou deixe em branco para não (\contemquadros{}).
\contemquadros{sim}

%Lista de siglas (opcional).
%Deseja incluir lista de abreviaturas e siglas? Caso afirmativo digite sim ou deixe em branco para não (\contemsiglas{}).
\contemsiglas{sim}

%Lista de símbolos (opcional).
%Deseja incluir lista de símbolos? Caso afirmativo digite sim ou deixe em branco para não (\contemsimbolos{}).
\contemsimbolos{sim}

%-------------------------FIM DOS DADOS BÁSICOS DO TCC--------------------------------------------




% ajusta espaçamento das listas itemize e enumerate
\setitemize{topsep=0pt,itemsep=0pt,leftmargin=\parindent+\labelwidth-\labelsep}
\setenumerate{topsep=0pt,itemsep=0pt,leftmargin=\parindent+\labelwidth-\labelsep}

% define a macro \Autoref to allow multiple references to be passed to \autoref
\makeatletter
\newcommand\Autoref[1]{\@first@ref#1,@}
\def\@throw@dot#1.#2@{#1}% discard everything after the dot
\def\@set@refname#1{%    % set \@refname to autoefname+s using \getrefbykeydefault
	\edef\@tmp{\getrefbykeydefault{#1}{anchor}{}}%
	\xdef\@tmp{\expandafter\@throw@dot\@tmp.@}%
	\ltx@IfUndefined{\@tmp autorefnameplural}%
	{\def\@refname{\@nameuse{\@tmp autorefname}s}}%
	{\def\@refname{\@nameuse{\@tmp autorefnameplural}}}%
}
\def\@first@ref#1,#2{%
	\ifx#2@\autoref{#1}\let\@nextref\@gobble% only one ref, revert to normal \autoref
	\else%
	\@set@refname{#1}%  set \@refname to autoref name
	\@refname~\ref{#1}% add autoefname and first reference
	\let\@nextref\@next@ref% push processing to \@next@ref
	\fi%
	\@nextref#2%
}
\def\@next@ref#1,#2{%
	\ifx#2@ e~\ref{#1}\let\@nextref\@gobble% at end: print e+\ref and stop
	\else, \ref{#1}% print  ,+\ref and continue
	\fi%
	\@nextref#2%
}
\makeatother

% Cria comando para referenciar Anexo automaticamente \refanexo
\newcommand{\refanexo}[1]{\hyperref[#1]{Anexo~\ref{#1}}}

% Define comandos para tabelas que permite ajustar o tamanho da coluna e manter alinhamento C, R ou L
%\newcommand{\PreserveBackslash}[1]{\let\temp=\\#1\let\\=\temp}
\newcolumntype{C}[1]{>{\centering\let\arraybackslash}m{#1}}
\newcolumntype{R}[1]{>{\RaggedLeft\let\arraybackslash}m{#1}}
\newcolumntype{L}[1]{>{\RaggedRight\let\arraybackslash}m{#1}}


% ---
% Filtering and Mapping Bibliographies
% ---
\DeclareSourcemap{
	\maps[datatype=bibtex]{
		% remove fields that are always useless
		\map{
			\step[fieldset=abstract, null]
			\step[fieldset=pagetotal, null]
			\step[fieldset=doi, null]
		}
		% remove URLs for types that are primarily printed
		\map{
			\pernottype{software}
			\pernottype{online}
			\pernottype{report}
			\pernottype{techreport}
			\pernottype{standard}
			\pernottype{manual}
			\pernottype{misc}
			\step[fieldset=url, null]
			\step[fieldset=urldate, null]
		}
		\map{
			\pertype{inproceedings}
			% remove mostly redundant conference information
			%\step[fieldset=venue, null]
			%\step[fieldset=eventdate, null]
			%\step[fieldset=eventtitle, null]
			% do not show ISBN for proceedings
			\step[fieldset=isbn, null]
			% Citavi bug
			%\step[fieldset=volume, null]
		}
	}
}
% ---

\preambulo
{%
	Trabalho de Conclusão de Curso de Graduação em Engenharia de Controle e Automação do Centro Tecnológico, de Ciências Exatas e Educação da Universidade Federal de Santa Catarina como requisito para a obtenção~do~título~de~\imprimirformacao.
}
% ---

% ---
% Configurações de aparência do PDF final
% ---
% alterando o aspecto da cor azul
\definecolor{blue}{RGB}{41,5,195}
% informações do PDF
\makeatletter
\hypersetup{
	%pagebackref=true,
	pdftitle={\@title}, 
	pdfauthor={\@author},
	pdfsubject={\imprimirpreambulo},
	pdfcreator={LaTeX with abnTeX2},
	pdfkeywords={ufsc, latex, abntex2}, 
	colorlinks=true,       		% false: boxed links; true: colored links
	linkcolor=black,%blue,          	% color of internal links
	citecolor=black,%blue,        		% color of links to bibliography
	filecolor=black,%magenta,      		% color of file links
	urlcolor=blue,
	bookmarksdepth=4
}
\makeatother
% ---

% Definição das siglas e símbolos

%----------------- LISTA DE ABREVIATURAS E SIGLAS--------------------------------------
\siglalista{INMET}{Instituto Nacional de Meteorologia}
\siglalista{SEMA-RS}{Secretaria do Meio Ambiente e Infraestrutura do Rio Grande do Sul}
\siglalista{MQO}{Mínimos Quadrados Ordinários}
\siglalista{RLS}{Regressão Linear Simples}
\siglalista{RLM}{Regressão Linear Múltipla}
\siglalista{ML}{Machine Learning (Aprendizado de Máquina)}
\siglalista{IQR}{Interquartile Range (Faixa Interquartílica)}
\siglalista{MSE}{Mean Squared Error (Erro Quadrático Médio)}
\siglalista{RMSE}{Root Mean Squared Error (Raiz do Erro Quadrático Médio)}
\siglalista{MAE}{Mean Absolute Error (Erro Absoluto Médio)}
\siglalista{MAPE}{Mean Absolute Percentage Error (Erro Percentual Absoluto Médio)}
\siglalista{R2}{Coeficiente de Determinação}
\siglalista{CSV}{Comma Separated Values (Valores Separados por Vírgula)}


%Para usar uma dada sigla ABC ao longo do texto, use \glsxtrfull{ABC} se quiser apresentar a sigla e sua definição.
%Se quiser apresentar apenas a sigla, use \gls{ABC}.






%-----------------SÍMBOLOS---------------------------------------------------------------
% \simbololista{C}{\ensuremath{C}}{Circunferência de um círculo}
% \simbololista{pi}{\ensuremath{\pi}}{Número pi} 
% \simbololista{r}{\ensuremath{r}}{Raio de um círculo}
% \simbololista{A}{\ensuremath{A}}{Área de um círculo}

%Para usar um dado símbolo SIMB ao longo do texto, use \gls{SIMB}.

% compila a lista de abreviaturas e siglas e a lista de símbolos
\makenoidxglossaries 

% compila o indice
\makeindex


% ------------------------------------------------------------------------------------------------
% --------------------------INÍCIO DO DOCUMENTO---------------------------------------------
% ------------------------------------------------------------------------------------------------
\begin{document}
	
	% Seleciona o idioma do documento (conforme pacotes do babel)
	%\selectlanguage{english}
	\selectlanguage{brazil}
	
	% Retira espaço extra obsoleto entre as frases.
	\frenchspacing 
	
	% Espaçamento 1.5 entre linhas
	\OnehalfSpacing
	
	% Corrige justificação
	%\sloppy
	

	%Elementos pré-textuais
	% \pretextual %a macro \pretextual é acionado automaticamente no início de \begin{document}
	% Capa, folha de rosto, ficha bibliografica, errata, folha de aprovação
	% Dedicatória, agradecimentos, epígrafe (opcional), resumos, listas
	\input{pre_textual/pretextual}

	% Elementos textuais
	\textual
	
	% 1 - Introdução
	% ----------------------------------------------------------
\chapter{Introdução}
% ----------------------------------------------------------

\section{Objetivos}

Nas seções abaixo estão descritos o objetivo geral e os objetivos específicos deste TCC.

\subsection{Objetivo Geral}

Descrição...

\subsection{Objetivos Específicos}

Descrição...
	
	% 2 - Desenvolvimento
	% ----------------------------------------------------------
\chapter{Fundamentação Teórica}\label{cap:fundamentacaoTeorica}
% ----------------------------------------------------------
Explicar brevemente o que será tratado como fundamentação teórica para o entendimento do contexto em que o modelo de aprendizagem de máquina será aplicado.

\section{A importância da cadeia de suprimentos}

Embora a globalização do mundo tenha tomado força no início do século XXI, intensificando a importação e exportação de matéria prima, commodities e produtos entre os países, a logística e o gerenciamento da cadeia de suprimentos são conceitos que impactam o desenvolvimento da humanidade há séculos. Desde a construção de pirâmides, até os esforços humanitários para a dimiuição da fome em países africanos se fundamentam no fluxo eficiente de materiais e insumos para que estes objetivos sejam alcançados.

Além disso, nas últimas guerras travadas pela humanidade, a capacidade logística foi um fator determinante para os países que saíram vitoriosos na história. Seja na estratégia para levar armamento ao território hostil, como também nos ataques a navios e comboios com comida e medicamentos para os soldados em combate, assegurar o planejamento e execução do transporte de materiais, e comprometer fluxo logístico do inimigo podia significar um passo adiante para a vitória na guerra. 

O Imperador Alexandre, o Grande, disse uma vez: "Meus especialistas em logística são muito sérios… pois sabem que, se minha campanha falhar, serão os primeiros que matarei". A afirmação mostra o quão importante a cadeia de suprimentos era para o império naquela época. Trazendo essa importância para a ótica dos dias atuais, basta olhar todos os móveis, itens de decoração, utensílios e ferramentas, seja do ambiente doméstico ou do trabalho, que se nota a necessidade dessa cadeia para tudo estar aonde está.

Mais do que um processo que está por trás da produção e envio de mercadorias ao redor do globo, a boa governança de uma cadeia de suprimentos constitui uma infraestrutura vital para o funcionamento e o desenvolvimento sustentável da sociedade.

\section{A gestão global da cadeia de suprimentos e seus desafios}

Segundo a empresa de tecnologia Totvs, "A cadeia de suprimentos (do inglês, \textit{supply chain}) é um sistema que envolve pessoas, processos e tecnologias focados em um objetivo: na melhor entrega possível de valor a um cliente, envolvendo todas as etapas de fabricação e entrega de produtos". 

Vale ressaltar a diferença entre a gestão logística e a gestão da cadeia de suprimentos (do inglês, \textit{Supply Chain management}, SCM), dado que a primeira está contida na segunda. Ao realizar toda essa gestão, estão incluídas todas as atividades que transformam matérias-primas em  produtos acabados para uso dos clientes, como o sourcing, design, produção, armazenamento, expedição e distribuição, ilustrado na Figura \ref{fig:Fig_1}

\begin{figure}[htb]
	\caption{\label{fig:Fig_1}Fluxo do SCM.}
	\begin{center}
		\includegraphics[scale=0.2]{figuras/what-is-supply-chain-management-graphic.png}
	\end{center}
	\fonte{Referência SAP}
\end{figure}

O gerenciamento da cadeia de suprimentos é essencial para a eficiência operacional e manutenção da competitividade de uma empresa no seu ramo de atuação. Um bom gerenciamento garante níveis de inventário que atendam às necessidades de produção, sem, ao mesmo tempo, gerar excessos que resultem em custos desnecessários ou em faltas de materiais que comprometam a capacidade de entrega no prazo. Tudo isso envolve um equilíbrio tênue entre a demanda prevista e a disponibilidade de suprimentos. 

Em condições normais de mercado, esse gerenciamento já é complexo, exigindo uma análise constante de dados de vendas, previsões de demanda e ajustes rápidos para evitar rupturas de estoque ou excessos. Sob condições atípicas, como as vivenciadas durante a pandemia, esse desafio se multiplica. A incerteza dos mercados, as restrições de fornecimento e as mudanças de comportamento do consumidor criam um cenário de alta volatilidade, exigindo um gerenciamento de estoque ainda mais dinâmico e ágil.

A crise dos semicondutores de 2021-2022 evidenciou as fragilidades das cadeias globais desse produto. Enquanto a demanda por chips disparou nesse período, a produção concentrada em poucos países, que ainda se recuperavam dos efeitos da pandemia, gerou escassez em diversas áreas da economia. O setor automotivo teve uma queda da produção de 15 milhões de veículos, e até mesmo o aprimoramento do acelerador de partículas LHC sofreu com adiamentos devido a falta de chips para o projeto. 

Através desses eventos, as empresas e nações evidenciaram a necessidade urgente da modernização dos processos de SCM, sendo mais flexíveis e resilientes a mudanças porém sem perder a estabilidade. Hoje, as melhores companhias analisam as operações dentro da área e suas tecnologias de execução, levantando questionamentos do que fazer para tornar os negócios mais eficientes, lucrativos e prontos para o futuro.

\section{Princípio de Pareto}

O Princípio de Pareto surgiu inicialmente no final do século XIX, quando o economista e sociólogo Vilfredo Pareto (1848-1923) notou que 80\% da riqueza de seu país (Itália) vinha de 20\% da população. Intrigado com a descoberta, Pareto aplicou a mesma lei a outros países como Rússia, França e Suíça, chegando ao mesmo resultado.

Mesmo após validar sua teoria em outros países, o Princípio de Pareto só foi reconhecido nos anos 40, pelo engenheiro americano Joseph Juran (1094-2008) ao aplicar a teoria na área da qualidade, e comprová-la em outras situações além das constatadas por Pareto.

De acordo com Antoine Delers, "o modelo provém da observação de que 20\% das causas são responsáveis por 80\% dos efeitos". 

Criar ligação do princípio de Pareto com as Classificações ABC-FMR

\subsection{Classificação ABC}

A classificação ABC é um método de categorização de materiais com base na importância relativa desses itens para uma organização. Os itens são classificados em três categorias principais - A, B e C - de acordo com seu valor ou impacto no desempenho geral da empresa.

\begin{itemize}
	\item Itens da Categoria A: agrega os itens de maior valor ou impacto, geralmente representando uma porcentagem significativa do valor total dos estoques ou das vendas. Esses itens são considerados críticos para o sucesso da empresa e requerem uma gestão mais rigorosa e atenção especial.
	\item Itens da Categoria B: classifica itens de valor moderado, que representam uma parte intermediária do valor total dos estoques ou das vendas. Embora não sejam tão críticos quanto os itens da Categoria A, os materiais dessa categoria ainda requerem um nível razoável de controle e monitoramento.
	\item Itens da Categoria C: itens de menor valor ou impacto, geralmente representando uma pequena parte do valor total dos estoques ou das vendas. Esses itens são considerados menos críticos e podem exigir menos atenção em comparação com os itens das categorias A e B.
\end{itemize}

Através da classificação ABC, a empresa dispõe de uma análise mais objetiva na gestão de estoques, compras e planejamento de produção, a fim de priorizar a alocação de recursos e esforços com base na importância relativa dos itens.

\subsection{Classificação FMR}

A classificação FMR é um método complementar à classificação ABC, utilizado para categorizar itens com base na frequência de demanda ou movimentação. A categorização dos materiais com base nessa classificação ajuda a determinar a estratégia de gestão de estoque mais adequada para cada item, considerando a sua movimentação no sistema.

\begin{itemize}
	\item Itens da categoria F (\textit{Fast Mover}): Itens com alta frequência de demanda ou movimentação. São produtos que têm uma alta rotatividade e são frequentemente solicitados pelos clientes. Esses itens geralmente requerem um estoque mais robusto para garantir disponibilidade imediata.
	\item Itens da categoria  M (\textit{Medium Mover}): Itens com uma frequência de demanda moderada. Embora não sejam tão solicitados quanto os itens da categoria F, esses produtos ainda têm uma demanda significativa e precisam ser gerenciados com atenção para evitar rupturas de estoque.
	\item Itens da categoria R (\textit{Rare Mover}): Itens com baixa frequência de demanda ou movimentação. São produtos que têm uma demanda esporádica ou sazonal, e geralmente não são solicitados com frequência. Esses itens podem não precisar de um estoque tão grande e podem ser gerenciados de forma mais flexível.
\end{itemize}

Além de classificar a frequência de demanda de um material, a classificação FMR também é útil para determinar se um item deve ser mantido em estoque (\textit{Make to Stock} - MTS) ou produzido sob demanda (\textit{Make to Order} - MTO). Desse modo, itens classificados como F podem ser mais adequados para a estratégia MTS, enquanto itens da categoria R são tratados para a estratégia MTO.

É a parte principal e mais extensa do trabalho. Deve apresentar a fundamentação teórica, a metodologia, os resultados e a discussão. Divide-se em seções e subseções conforme a NBR 6024 \cite{NBR6024:2012}.

Quanto à sua estrutura e projeto gráfico, segue as recomendações da norma para preparação de trabalhos acadêmicos, a NBR 14724, de 2011 \cite{NBR14724:2011}.

\begin{figure}[htb]
	\caption{\label{fig:Fig_2}Elementos do trabalho acadêmico.}
	\begin{center}
		\includegraphics{figuras/imagem.pdf}
	\end{center}
	\fonte{Universidade Federal do Paraná (1996).}
\end{figure}

\subsection{Formatação do texto}

No que diz respeito à estrutura do trabalho, recomenda-se que:
\begin{alineas}
	\item o texto deve ser justificado, digitado em cor preta, podendo utilizar outras cores somente para as ilustrações;
	\item utilizar papel branco ou reciclado para impressão;
	\item \textbf{se o trabalho for impresso}, os elementos pré-textuais devem iniciar no anverso da folha, com exceção da ficha catalográfica ou ficha de identificação da obra;
	\item \textbf{se o trabalho for impresso}, os elementos textuais e pós-textuais devem ser digitados no anverso e verso das folhas;
	\item as seções primárias devem começar sempre em páginas ímpares, quando o trabalho for impresso e
	\item deixar um espaço entre o título da seção/subseção e o texto e entre o texto e o título da subseção.
\end{alineas}

No \autoref{qua:Quadro_1} estão as especificações para a formatação do texto.

\begin{quadro}[htb]
	\centering
	\caption{\label{qua:Quadro_1}Formatação do texto.}	
	\begin{tabular}{|l|p{11cm}|}
		\hline
		\textbf{Formato do papel} & A4.\\ \hline
		\textbf{Impressão}        & A norma recomenda que \textbf{caso seja necessário imprimir}, deve-se utilizar a frente e o verso da página.\\ \hline
		\textbf{Margens}          & Superior: 3, Inferior: 2, Interna: 3 e Externa: 2. Usar margens espelhadas quando o  trabalho for impresso.\\ \hline
		\textbf{Paginação}        & As páginas dos elementos pré-textuais devem ser contadas, mas não numeradas. Para trabalhos digitados somente no anverso, a numeração das páginas deve constar no canto superior direito da página, a 2 cm da borda, figurando a partir da primeira folha da  parte textual. Para trabalhos digitados no anverso e no verso, a numeração deve constar no canto superior direito, no anverso, e no canto superior esquerdo no verso.\\ \hline
		\textbf{Espaçamento}      & O texto deve ser redigido com espaçamento entre linhas 1,5, excetuando-se as citações de mais de três linhas, notas de rodapé, referências, legendas das ilustrações e das tabelas, natureza (tipo do trabalho, objetivo, nome da instituição a que é submetido e área de concentração), que devem ser digitados em espaço simples, com fonte menor. As referências devem ser separadas entre si por um espaço simples em branco.\\ \hline
		\textbf{Paginação}        & A contagem inicia na folha de rosto, mas se \textbf{insere o número da página na introdução} até o final do trabalho.\\ \hline
		\textbf{Fontes sugeridas} & Arial ou Times New Roman.\\ \hline
		\textbf{Tamanho da fonte} & \textbf{Fonte tamanho 12 para o texto}, incluindo os títulos das seções e subseções. As citações com mais de três linhas, notas de rodapé, paginação, dados internacionais de catalogação, legendas e fontes das ilustrações e das tabelas devem ser de tamanho menor. Adotamos, neste \textit{template} \textbf{fonte tamanho 10}.\\ \hline
		\textbf{Nota de rodapé}   & Devem ser digitadas dentro da margem, ficando separadas por um espaço simples por entre as linhas e por filete de 5 cm a partir da margem esquerda. A partir da segunda linha, devem ser alinhadas embaixo da primeira letra da primeira palavra da primeira linha.\\ \hline
	\end{tabular}
	\fonte{\textcite{NBR14724:2011}.}
\end{quadro}


\subsubsection{As ilustrações}

Independentemente do tipo de ilustração (quadro, desenho, figura, fotografia, mapa, entre outros), a sua identificação aparece na parte superior, precedida da palavra designativa. 

\begin{citacao}
	Após a ilustração, na parte inferior, indicar a fonte consultada (elemento obrigatório, mesmo que seja produção do próprio autor), legenda, notas e outras informações necessárias à sua compreensão (se houver). A ilustração deve ser citada no texto e inserida o mais próximo possível do texto a que se refere. \cite[p. 11]{NBR14724:2011}.
\end{citacao}

\subsubsection{Equações e fórmulas}

As equações e fórmulas devem ser destacadas no texto para facilitar a leitura.  Para numerá-las, usar algarismos arábicos entre parênteses e alinhados à direita. Pode-se adotar uma entrelinha maior do que a usada no texto \cite{NBR14724:2011}.

Por exemplo, a circunferência e a área de um círculo com raio $r$ são dados, respectivamente, por
\begin{equation}\label{eq:Eq_1}
\gls{C} = 2 \gls{pi} \gls{r}    %note que o comando \gls{} usa a definição da lista de símbolos. Se não houver lista de símbolos, a equação deve ser digitada normalmente, ou seja, C = 2 \pi r
\end{equation}
e
\begin{equation}\label{eq:Eq_2}
\gls{A} = \gls{pi} \gls{r}^2.
\end{equation}
É importante observar que a \autoref{eq:Eq_1} e a \autoref{eq:Eq_2} fazem parte da frase (note a letra ``e'' entre as equações e o ponto final após a \autoref{eq:Eq_2}). 

\subsubsubsection{Exemplo tabela}

De acordo com \textcite{ibge1993}, tabela é uma forma não discursiva de apresentar informações em que os números representam a informação central. Ver \autoref{tab:Tab_1}.

\begin{table}[htb]
	\ABNTEXfontereduzida
	\caption{\label{tab:Tab_1}Médias concentrações urbanas 2010-2011.}
	\begin{tabular}{@{}p{3.0cm}p{1.5cm}p{2cm}p{2.5cm}p{2.5cm}p{2.5cm}@{}}
		\toprule
		\textbf{Média concentração urbana} & \multicolumn{2}{l}{\textbf{População}} & \textbf{Produto Interno Bruto – PIB (bilhões R\$)} & \textbf{Número de empresas} & \textbf{Número de unidades locais} \\ \midrule
		\textbf{Nome}                      & \textbf{Total}   & \textbf{No Brasil}  &                                                   &                             & \\
		Ji-Paraná (RO)                     & 116 610          & 116 610             & 1,686                                             & 2 734                       & 3 082 \\
		Parintins (AM)                     & 102 033          & 102 033             & 0,675                                             & 634                         & 683 \\
		Boa Vista (RR)                     & 298 215          & 298 215             & 4,823                                             & 4 852                       & 5 187 \\
		Bragança (PA)                      & 113 227          & 113 227             & 0,452                                             & 654                         & 686 \\ \bottomrule
	\end{tabular}
	\fonte{\textcite{ibge2016}.}
\end{table}
	
	% 3 - Seção
	% ----------------------------------------------------------
\chapter{Preparação dos dados para o treinamento}
% ----------------------------------------------------------

Antes de iniciar o treinamento, exitem alguns procedimentos iniciais que garantem desde a coleta correta das informações necessárias para o modelo, até a limpeza, transformação e redução dos dados. Nesta seção, serão abordadas as etapas necessárias para preparar os dados para o treinamento do modelo, seguindo o fluxo de trabalho descrito na Figura \ref{fig:passos_preparacao}.

\begin{figure}[H]
	\caption{\label{fig:passos_preparacao}Passos para preparação dos dados.}
	\begin{center}
		\includegraphics[scale=0.4]{figuras/steps_data_preparing.png}
	\end{center}
	\fonte{\cite{pecan_data_prep_2023}}
\end{figure}

\section{Coleta de dados}

Considerando a premissa do trabalho, em que a previsão do nível do rio será dada a partir de dados meteorológicos da cidade de Porto Alegre, junto aos dados de monitoramento do nível dos rios que constituem a bacia do Guaíba, duas fonte de dados foram utilizadas. Para os dados meteorológicos, o portal do \textit{INMET}\footnote{\url{https://portal.inmet.gov.br/}} (Instituto Nacional de Meteorologia) foi utilizado, onde foram coletadas as informações de temperatura, umidade relativa do ar, precipitação e velocidade do vento. Já os dados de monitoramento dos rios foram coletados no site da \textit{SEMA-RS}\footnote{\url{https://www.saladesituacao.rs.gov.br/dados}} (Sala de situação), onde foram coletados os dados de nível do rio Guaíba,Caí, Jacuí, Sinos e Gravataí. Os dados meteorológicos foram coletados em formato \textit{CSV}, com arquivos separados em anos de monitoramento com frequência horária, enquanto os dados dos níveis dos rios foram coletados no formato \textit{.xlsx}, com histórico completo de amostragem das informações em frequência de 15 minutos, utilizando a biblioteca \textit{Pandas} do Python.

\section{Pré processamento dos dados}

Antes de seguir para o próximo passo mostrado na Figura \ref{fig:passos_preparacao}, os dados coletados necessitam de um pré-processamento específico para cada uma das fontes utilizadas. Para os dados meteorológicos, devido às informações estarem separadas por ano de monitoramento, foi necessário concatenar os arquivos de cada ano em um único arquivo, utilizando a função \textit{concat} da biblioteca \textit{Pandas}, removendo o cabeçalho de informações geográficas da estação, ilustrado na Figura \ref{fig:base_inmet} Além disso, foi necessário combinar as duas primeiras colunas e converter o formato de data e hora para o padrão \textit{datetime}, utilizando a função \textit{to\_datetime} da mesma biblioteca. 

\begin{figure}[H]
	\caption{\label{fig:base_inmet}Dados meteorológicos.}
	\begin{center}
		\includegraphics[scale=0.25]{figuras/base_inmet.png}
	\end{center}
	\fonte{Autor.}
\end{figure}

Analisando as colunas e os seus respectivos tipos de dados, tem-se a seguinte tabela:

\begin{table}[H]
	\centering
	\begin{tabular}{|p{10cm}|c|}
	\hline
	\textbf{Coluna} & \textbf{Tipo} \\
	\hline
	Precipitação Total (mm) & float64 \\
	Pressão Atmosférica ao Nível da Estação (mB) & float64 \\
	Pressão Atmosférica Máx. na Hora Anterior (mB) & float64 \\
	Pressão Atmosférica Mín. na Hora Anterior (mB) & float64 \\
	Radiação Global (kJ/m²) & float64 \\
	Temperatura do Ar - Bulbo Seco (°C) & float64 \\
	Temperatura do Ponto de Orvalho (°C) & object \\
	Temperatura Máxima na Hora Anterior (°C) & object \\
	Temperatura Mínima na Hora Anterior (°C) & object \\
	Temperatura Orvalho Máx. na Hora Anterior (°C) & float64 \\
	Temperatura Orvalho Mín. na Hora Anterior (°C) & float64 \\
	Umidade Relativa Máx. na Hora Anterior (\%) & float64 \\
	Umidade Relativa Mín. na Hora Anterior (\%) & object \\
	Umidade Relativa do Ar (\%) & object \\
	Vento - Direção Horária (° (gr)) & object \\
	Vento - Rajada Máxima (m/s) & float64 \\
	Vento - Velocidade Horária (m/s) & float64 \\
	\hline
	\end{tabular}
	\caption{Tabela de tipos de dados da base de informações meteorológicas.}
	\label{tab:colunas_dados_meteorologicos}
\end{table}

A partir dela, nota-se que algumas colunas estão com o tipo de dado \textit{object}, o que indica que os dados não estão no formato correto. Para resolver isso, foi necessário converter as colunas de temperatura e umidade relativa do ar para o tipo \textit{float64}.

Já para os dados dos níveis dos rios, também foi necessário remover o cabeçalho com dados geográficos da estação de monitoramento, como mostra a Figura \ref{fig:base_sema}, junto da conversão do formato de data e hora para o padrão \textit{datetime} da primeira coluna.

\begin{figure}[H]
	\caption{\label{fig:base_sema}Dados meteorológicos.}
	\begin{center}
		\includegraphics[scale=0.5]{figuras/base_sema.png}
	\end{center}
	\fonte{Autor.}
\end{figure}

\section{Limpeza dos dados}

Após a coleta dos dados, o próximo passo é a limpeza dessas informações, cujo procedimento consiste em remover dados duplicados, corrigir erros de formatação e lidar com valores ausentes. Existem diferentes abordagens para tratar valores inconsistentes no dados coletados. Uma abordagem comum, principalmente em casos onde não há uma linearidade ou tendência clara é o preenchimento com zero, como mostra a Figura \ref{fig:passo_dados_limpeza}, ou com a média dos dados na coluna a ser limpa.

\begin{figure}[H]
	\caption{\label{fig:passo_dados_limpeza}Limpeza dos dados coletados.}
	\begin{center}
		\includegraphics[scale=0.4]{figuras/step_data_cleaning.png}
	\end{center}
	\fonte{\cite{pecan_data_prep_2023}}
\end{figure}

Nessa etapa, na base de dados meteorológicos, optou-se por preencher os dados ausentes, representados por ''-'', e os dados com valor igual a ''-9999'' por zero, já que dados meteorológicos tendem a ser mais voláteis e não apresentam uma tendência clara. Desse modo, a abordagem de aproximação linear é descartada, a fim de não comprometer a análise do modelo de previsão.  

Tomando como exemplo a coluna \textit{TEMPERATURA ORVALHO MAX. NA HORA ANT. (AUT) (°C)}, listada na Tabela \ref{tab:colunas_dados_meteorologicos}, na Figura \ref{fig:dados_clima_poa_nao_tratados}, observa-se os valores ausentes representados por ''-'' e ''-9999''.

\begin{figure}[H]
	\caption{\label{fig:dados_clima_poa_nao_tratados}Gráfico de Temperatura do orvalho Máx. na Hora Anterior (°C) não tratado.}
	\begin{center}
		\includegraphics[scale=0.35]{figuras/TEMPERATURA ORVALHO MAX. NA HORA ANT. (AUT) (°C) SEM TRATAMENTO.png}
	\end{center}
	\fonte{Autor.}
\end{figure}

Após a substituição dos valores ausentes por zero, o gráfico da mesma coluna, mostrado na Figura \ref{fig:dados_clima_poa_tratados}, apresenta uma distribuição mais uniforme, sem os picos de dados ausentes.

\begin{figure}[H]
	\caption{\label{fig:dados_clima_poa_tratados}Gráfico de Temperatura do orvalho Máx. na Hora Anterior (°C) tratado.}
	\begin{center}
		\includegraphics[scale=0.35]{figuras/TEMPERATURA ORVALHO MAX. NA HORA ANT. (AUT) (°C) COM TRATAMENTO.png}
	\end{center}
	\fonte{Autor.}
\end{figure}

Aplicado o tratamento, a coluna \textit{TEMPERATURA ORVALHO MAX. NA HORA ANT. (AUT) (°C)} passa a apresentar uma distribuição mais uniforme, sem os picos de dados ausentes, com valores condizentes com a escala esperada da unidade medida, neste caso, a temperatura em graus Celsius.

Em casos onde os dados apresentam \textit{outliers} (dados que se distanciam significativamente do restante da coluna analisada), foi aplicado o método IQR (Interquartile Range) para identificar e remover esses valores. A partir da diferença entre o terceiro quartil (Q3) e o primeiro quartil (Q1) de um conjunto de dados, multiplicado por um fator de tolerância, o método determina limites superiores e inferiores para o conjunto de dados analisado, removendo ou substituindo os valores que estão fora dessa faixa, para manter a consistência das informações.

Por exemplo, na coluna \textit{UMIDADE REL. MAX. NA HORA ANT. (AUT) (\%)}, as Figuras \ref{fig:dados_outlier_sem_tratamento} e \ref{fig:dados_outlier_com_tratamento} mostram a diferença entre os dados após a aplicação do método IQR, com um fator de tolerância de 1.2 entre os interquartis Q1 e Q3.

\begin{figure}[H]
	\caption{\label{fig:dados_outlier_sem_tratamento}Gráfico de Umidade Relativa Máxima  na Hora Anterior (\%) sem tratamento.}
	\begin{center}
		\includegraphics[scale=0.35]{figuras/UMIDADE REL. MAX. NA HORA ANT. (AUT) SEM TRATAMENTO.png}
	\end{center}
	\fonte{Autor.}
\end{figure}

\begin{figure}[H]
	\caption{\label{fig:dados_outlier_com_tratamento}Gráfico de Umidade Relativa Máxima  na Hora Anterior (\%) com tratamento.}
	\begin{center}
		\includegraphics[scale=0.35]{figuras/UMIDADE REL. MAX. NA HORA ANT. (AUT) COM TRATAMENTO.png}
	\end{center}
	\fonte{Autor.}
\end{figure}

Na base de dados dos níveis dos rios, embora a dinâmica que representa os respectivos comportamentos não seja linear, o uso do conceito de aproximação linear permite preencher os dados ausentes por meio da interpolação entre os valores que cercam as células sem informação em determinado período. Essa abordagem é válida, uma vez que a variação do nível do rio entre dois pontos de amostragem próximos tende a ser linear, já que o nível do rio não apresenta oscilações bruscas em curtos períodos de tempo.

Para realizar a interpolação, utilizou-se a função \textit{interpolate} com o argumento \textit{method = linear} da biblioteca \textit{Pandas}, que preenche os valores ausentes com base nos dados adjacentes.

Ainda, observou-se nas bases dos níveis dos rios a ocorrência de valores ausentes no início ou no final do período de amostragem, inviabilizando a interpolação. Para esses casos, o preenchimento foi feito a partir da repetição do primeiro ou do último valor disponível, respectivamente. É importante ressaltar que tal limpeza só pode ser feita se o preenchimento ocorra em um intervalo de tempo curto, visando não afetar a análise feita pelo modelo de previsão.

\section{Redução dos dados}

A etapa de redução dos dados visa eliminar informações redundantes ou irrelevantes, mantendo apenas os dados que contribuem para a análise e treinamento do modelo de previsão. Considerando que serão utilizados dados de monitoramento do nível de 4 rios para a previsão do nível do rio Guaíba, os dados meteorológicos coletados precisam ser filtrados de modo a garantir tanto que as informações sejam relevantes para a previsão, quanto que os dados de monitoramento dos rios estejam alinhados com os dados meteorológicos, evitando assim a inclusão de dados desnecessários no treinamento. 

Voltando para a Tabela \ref{tab:colunas_dados_meteorologicos}, nota-se a presença de 17 colunas, das quais:

\begin{itemize}
	\item 6 colunas referem-se a dados de temperatura;
	\item 3 colunas referem-se a dados de pressão atmosférica;
	\item 3 colunas referem-se a dados do comportamento do vento;
	\item 3 colunas referem-se a dados de umidade relativa do ar;
	\item 1 coluna refere-se a dados de radiação solar.
	\item 1 coluna refere-se a dados de precipitação.
\end{itemize}

Desse modo, para o primeiro passo de redução da base, foram mantidas apenas uma coluna de cada tipo de dado, filtrando 6 colunas no total, conforme a Tabela \ref{tab:colunas_dados_meteorologicos_reduzidas}. 

\begin{table}[H]
	\centering
	\begin{tabular}{|p{10cm}|c|}
	\hline
	\textbf{Coluna} & \textbf{Unidade de medida} \\
	\hline
	Temperatura do Ar - Bulboeco & (°C) \\
	Pressão Atmosférica ao Nível da Estação & (mB) \\
	Vento - Velocidade Horária & (m/s) \\
	Umidade Relativa do Ar & (\%) \\
	Radiação Global & (kJ/m²) \\
	Precipitação Total & (mm) \\
	\hline
	\end{tabular}
	\caption{Tabela de dados meteorológicos reduzidos - 1ª Filtragem}
	\label{tab:colunas_dados_meteorologicos_reduzidas}
\end{table}

Para a escolha das colunas a serem mantidas, foram levados em consideração os seguintes critérios:
\begin{itemize}
	\item As colunas \textit{Temperatura do Ar - Bulboeco}, \textit{Pressão Atmosférica ao Nível da Estação} e \textit{Umidade Relativa do Ar} foram escolhidas por serem medidas diretas dos respectivos fenômenos atmosféricos, enquanto as demais colunas referem-se a medidas derivadas ou não diretamente observáveis, que não são tão relevantes para a previsão do nível do rio.
	\item A coluna \textit{Precipitação Total} foi mantida por ser um dos principais fatores que influenciam o nível do rio.
	\item As colunas \textit{Radiação Global} e \textit{Umidade Relativa do Ar} foram mantidas por serem fatores importantes para a evaporação da água, que também influenciam, mesmo que de forma indireta, no nível do rio.
\end{itemize}

Após essa redução inicial, os gráficos e seus dados de cada coluna foram analisados, a fim de verificar se as informações eram consistentes e poderiam contribuir para o treinamento do modelo. Para isso, foram utilizados gráficos de dispersão e histogramas, como os mostrados na Figura \ref{fig:histograma_dados_meteorologicos}.

ADICIONAR AQUI O HISTOGRAMA DOS DADOS METEOROLÓGICOS E A DECISÃO DE QUAIS COLUNAS FORAM MANTIDAS E QUAIS FORAM DESCARTADAS.

Definidas as colunas a serem mantidas, o próximo passo consistiu em alinhar os dados meteorológicos com os dados de monitoramento dos rios, garantindo que as informações estivessem na mesma frequência de amostragem. Para isso, foram utilizados os dados de monitoramento do nível do rio Guaíba, que possuem a maior frequência de amostragem entre os rios analisados, com intervalo de 15 minutos, enquanto os dados meteorológicos possuem frequência de 1 hora (60 minutos). Assim, a frequência de amostragem das informações do nível dos rios foi reduzida para 1 hora, aplicando um filtro simples que mantém apnas as linhas cujo \textit{timestamp} tem minuto igual a 0.


\section{Transformação dos dados}




	
	% 4 - Conclusão
	% ----------------------------------------------------------
\chapter{Conclusão}
% ----------------------------------------------------------
O presente trabalho teve como objetivo geral desenvolver um modelo de previsão do nível do Rio Guaíba, utilizando dados meteorológicos por meio da técnica de Regressão Ridge. Para alcançar esse propósito, foram estabelecidos objetivos específicos que envolveram desde a coleta e pré-processamento dos dados, até a implementação do modelo e análise dos resultados.

No desenvolvimento do projeto, procedeu-se à coleta de dados meteorológicos disponibilizados pelo INMET, assim como dos níveis hidrométricos dos rios da bacia do Guaíba, através do SEMA-RS. Esses dados passaram por um pré-processamento, que incluiu a limpeza, tratamento de valores ausentes, remoção de outliers e normalização, garantindo maior qualidade ao conjunto utilizado para treinamento. Foi também realizada a redução da dimensionalidade, mantendo apenas variáveis com maior relevância para a previsão, como temperatura do ar, pressão atmosférica, radiação global, velocidade do vento, umidade relativa e precipitação.

Na etapa de implementação, o modelo de Regressão Ridge foi escolhido por sua capacidade de lidar com multicolinearidade e reduzir o risco de sobreajuste, característica importante considerando a volatilidade das variáveis climáticas. Foram testados diferentes valores do parâmetro de regularização (\texttt{alpha}), variando também as proporções de divisão entre conjuntos de treinamento e teste (80:20, 70:30, 60:40). Para avaliar o desempenho da previsão do modelo, as métricas de desempenho — MSE, RMSE, MAE e R² foram utilizadas, visando uma análise mais objetiva e estatística da implementação.

Os resultados evidenciaram a robustez do modelo e a correlação significativa entre variáveis meteorológicas e o nível do Rio Guaíba. Apesar de pequenas variações nos indicadores de desempenho, não se observou impacto expressivo da alteração dos valores de alpha, sugerindo que o modelo atinge um platô de desempenho para os intervalos testados. Tal constatação indica que ajustes adicionais de regularização, ou a inclusão de novas variáveis externas, poderiam ser explorados em trabalhos futuros para ganhos incrementais de performance. Além disso, cabe a comparação com outros modelos de previsão existentes, ou até mesmo uma rede neural para comparar o desempenho em relação ao modelo Ridge.

Assim, conclui-se que o objetivo proposto foi atingido. O modelo desenvolvido mostrou-se eficiente e tecnicamente sólido para prever o nível do Rio Guaíba a partir dos dados meteorológicos utilizados. Entretanto, vale ressaltar que os dados utilizados são históricos, ou seja, compõem medidas meteorológicas aferidas por instrumentos no momento da medição. Para uma aplicação real de previsão do nível do rio, visando um alerta antecipado de enchente, o modelo seria expostos a medidas meteorológicas previstas ao invés de medidas, adicionando um grau de incerteza que pode gerar previsões com um grau de erro maior. Por isso, é necessário realizar testes com dados previstos, avaliar novas métricas de desempenho, para então considerar o uso do modelo em casos reais de previsão do nível do rio.  
	
	
	% Elementos pós-textuais
	\postextual
	
	
	% Referências bibliográficas
	\begingroup
	    \printbibliography[title=REFERÊNCIAS]
	\endgroup
	
	
	
	%Reconfiguração do título para apêndices e anexos
	 \renewcommand{\ABNTEXchapterupperifneeded}[1]{#1} 
	\makeatletter
	\settocpreprocessor{chapter}{%
      \let\tempf@rtoc\f@rtoc%
      \def\f@rtoc{%
      \texorpdfstring{{\tempf@rtoc}}{\tempf@rtoc}}%
      }
    \makeatother
	
	
	% Apêndices
    \begin{apendicesenv}
    	%\partapendices* 
    	\input{pos_textual/apendice_a}
    \end{apendicesenv}

    % Anexos
    \begin{anexosenv}
    	%\partanexos*
    	\input{pos_textual/anexo_a}
    \end{anexosenv}

\end{document}